% Options for packages loaded elsewhere
\PassOptionsToPackage{unicode}{hyperref}
\PassOptionsToPackage{hyphens}{url}
%
\documentclass[
]{article}
\usepackage{amsmath,amssymb}
\usepackage{lmodern}
\usepackage{iftex}
\ifPDFTeX
  \usepackage[T1]{fontenc}
  \usepackage[utf8]{inputenc}
  \usepackage{textcomp} % provide euro and other symbols
\else % if luatex or xetex
  \usepackage{unicode-math}
  \defaultfontfeatures{Scale=MatchLowercase}
  \defaultfontfeatures[\rmfamily]{Ligatures=TeX,Scale=1}
\fi
% Use upquote if available, for straight quotes in verbatim environments
\IfFileExists{upquote.sty}{\usepackage{upquote}}{}
\IfFileExists{microtype.sty}{% use microtype if available
  \usepackage[]{microtype}
  \UseMicrotypeSet[protrusion]{basicmath} % disable protrusion for tt fonts
}{}
\makeatletter
\@ifundefined{KOMAClassName}{% if non-KOMA class
  \IfFileExists{parskip.sty}{%
    \usepackage{parskip}
  }{% else
    \setlength{\parindent}{0pt}
    \setlength{\parskip}{6pt plus 2pt minus 1pt}}
}{% if KOMA class
  \KOMAoptions{parskip=half}}
\makeatother
\usepackage{xcolor}
\usepackage[margin=1in]{geometry}
\usepackage{color}
\usepackage{fancyvrb}
\newcommand{\VerbBar}{|}
\newcommand{\VERB}{\Verb[commandchars=\\\{\}]}
\DefineVerbatimEnvironment{Highlighting}{Verbatim}{commandchars=\\\{\}}
% Add ',fontsize=\small' for more characters per line
\usepackage{framed}
\definecolor{shadecolor}{RGB}{248,248,248}
\newenvironment{Shaded}{\begin{snugshade}}{\end{snugshade}}
\newcommand{\AlertTok}[1]{\textcolor[rgb]{0.94,0.16,0.16}{#1}}
\newcommand{\AnnotationTok}[1]{\textcolor[rgb]{0.56,0.35,0.01}{\textbf{\textit{#1}}}}
\newcommand{\AttributeTok}[1]{\textcolor[rgb]{0.77,0.63,0.00}{#1}}
\newcommand{\BaseNTok}[1]{\textcolor[rgb]{0.00,0.00,0.81}{#1}}
\newcommand{\BuiltInTok}[1]{#1}
\newcommand{\CharTok}[1]{\textcolor[rgb]{0.31,0.60,0.02}{#1}}
\newcommand{\CommentTok}[1]{\textcolor[rgb]{0.56,0.35,0.01}{\textit{#1}}}
\newcommand{\CommentVarTok}[1]{\textcolor[rgb]{0.56,0.35,0.01}{\textbf{\textit{#1}}}}
\newcommand{\ConstantTok}[1]{\textcolor[rgb]{0.00,0.00,0.00}{#1}}
\newcommand{\ControlFlowTok}[1]{\textcolor[rgb]{0.13,0.29,0.53}{\textbf{#1}}}
\newcommand{\DataTypeTok}[1]{\textcolor[rgb]{0.13,0.29,0.53}{#1}}
\newcommand{\DecValTok}[1]{\textcolor[rgb]{0.00,0.00,0.81}{#1}}
\newcommand{\DocumentationTok}[1]{\textcolor[rgb]{0.56,0.35,0.01}{\textbf{\textit{#1}}}}
\newcommand{\ErrorTok}[1]{\textcolor[rgb]{0.64,0.00,0.00}{\textbf{#1}}}
\newcommand{\ExtensionTok}[1]{#1}
\newcommand{\FloatTok}[1]{\textcolor[rgb]{0.00,0.00,0.81}{#1}}
\newcommand{\FunctionTok}[1]{\textcolor[rgb]{0.00,0.00,0.00}{#1}}
\newcommand{\ImportTok}[1]{#1}
\newcommand{\InformationTok}[1]{\textcolor[rgb]{0.56,0.35,0.01}{\textbf{\textit{#1}}}}
\newcommand{\KeywordTok}[1]{\textcolor[rgb]{0.13,0.29,0.53}{\textbf{#1}}}
\newcommand{\NormalTok}[1]{#1}
\newcommand{\OperatorTok}[1]{\textcolor[rgb]{0.81,0.36,0.00}{\textbf{#1}}}
\newcommand{\OtherTok}[1]{\textcolor[rgb]{0.56,0.35,0.01}{#1}}
\newcommand{\PreprocessorTok}[1]{\textcolor[rgb]{0.56,0.35,0.01}{\textit{#1}}}
\newcommand{\RegionMarkerTok}[1]{#1}
\newcommand{\SpecialCharTok}[1]{\textcolor[rgb]{0.00,0.00,0.00}{#1}}
\newcommand{\SpecialStringTok}[1]{\textcolor[rgb]{0.31,0.60,0.02}{#1}}
\newcommand{\StringTok}[1]{\textcolor[rgb]{0.31,0.60,0.02}{#1}}
\newcommand{\VariableTok}[1]{\textcolor[rgb]{0.00,0.00,0.00}{#1}}
\newcommand{\VerbatimStringTok}[1]{\textcolor[rgb]{0.31,0.60,0.02}{#1}}
\newcommand{\WarningTok}[1]{\textcolor[rgb]{0.56,0.35,0.01}{\textbf{\textit{#1}}}}
\usepackage{graphicx}
\makeatletter
\def\maxwidth{\ifdim\Gin@nat@width>\linewidth\linewidth\else\Gin@nat@width\fi}
\def\maxheight{\ifdim\Gin@nat@height>\textheight\textheight\else\Gin@nat@height\fi}
\makeatother
% Scale images if necessary, so that they will not overflow the page
% margins by default, and it is still possible to overwrite the defaults
% using explicit options in \includegraphics[width, height, ...]{}
\setkeys{Gin}{width=\maxwidth,height=\maxheight,keepaspectratio}
% Set default figure placement to htbp
\makeatletter
\def\fps@figure{htbp}
\makeatother
\setlength{\emergencystretch}{3em} % prevent overfull lines
\providecommand{\tightlist}{%
  \setlength{\itemsep}{0pt}\setlength{\parskip}{0pt}}
\setcounter{secnumdepth}{-\maxdimen} % remove section numbering
\ifLuaTeX
  \usepackage{selnolig}  % disable illegal ligatures
\fi
\IfFileExists{bookmark.sty}{\usepackage{bookmark}}{\usepackage{hyperref}}
\IfFileExists{xurl.sty}{\usepackage{xurl}}{} % add URL line breaks if available
\urlstyle{same} % disable monospaced font for URLs
\hypersetup{
  pdftitle={Getting started with R and RStudio},
  hidelinks,
  pdfcreator={LaTeX via pandoc}}

\title{Getting started with R and RStudio}
\author{}
\date{\vspace{-2.5em}}

\begin{document}
\maketitle

\begin{Shaded}
\begin{Highlighting}[]
\FunctionTok{library}\NormalTok{(tidyverse)}
\end{Highlighting}
\end{Shaded}

\begin{verbatim}
## -- Attaching packages --------------------------------------- tidyverse 1.3.2 --
## v ggplot2 3.4.0      v purrr   0.3.5 
## v tibble  3.1.8      v dplyr   1.0.10
## v tidyr   1.2.1      v stringr 1.5.0 
## v readr   2.1.3      v forcats 0.5.2 
## -- Conflicts ------------------------------------------ tidyverse_conflicts() --
## x dplyr::filter() masks stats::filter()
## x dplyr::lag()    masks stats::lag()
\end{verbatim}

\hypertarget{r-markdown}{%
\subsection{R Markdown}\label{r-markdown}}

This is an \href{http://rmarkdown.rstudio.com}{R Markdown} file (it has
a .Rmd file extension). When you execute code within the file, the
results appear beneath the code.

R code goes in \textbf{code chunks}, denoted by three backticks. Try
executing this chunk by clicking the \emph{Run} button (a small green
triangle) within the chunk or by placing your cursor inside it and
pressing \emph{Ctrl+Shift+Enter} (or \emph{Cmd+Shift+Enter} on Mac).

\begin{Shaded}
\begin{Highlighting}[]
\FunctionTok{ggplot}\NormalTok{(}\AttributeTok{data =}\NormalTok{ mpg) }\SpecialCharTok{+}
  \FunctionTok{geom\_point}\NormalTok{(}\AttributeTok{mapping =} \FunctionTok{aes}\NormalTok{(}\AttributeTok{x =}\NormalTok{ cty, }\AttributeTok{y =}\NormalTok{ hwy), }\AttributeTok{alpha =} \FloatTok{0.2}\NormalTok{)}
\end{Highlighting}
\end{Shaded}

\includegraphics{01_getting-started_files/figure-latex/unnamed-chunk-1-1.pdf}

\hypertarget{add-a-new-code-chunk}{%
\subsection{Add a new code chunk}\label{add-a-new-code-chunk}}

Add a new code chunk by clicking the \emph{Insert Chunk} button on the
toolbar or by pressing \emph{Cmd/Ctrl+Option+I}.

Put 2 + 2 in your new code chunk and run it.

2 + 2

\begin{Shaded}
\begin{Highlighting}[]
\DecValTok{2} \SpecialCharTok{+} \DecValTok{2}
\end{Highlighting}
\end{Shaded}

\begin{verbatim}
## [1] 4
\end{verbatim}

\hypertarget{knitting-r-markdown-files}{%
\subsection{Knitting R Markdown files}\label{knitting-r-markdown-files}}

We'll use R Markdown files as notebooks as we learn because we can
record text, code and output.

R Markdown files are also a publication format. Try hitting the ``Knit''
button in the toolbar above. R runs all the code in the document from
top to bottom, it collects the output and puts the code, text and output
together in an HTML document---you should see it as
\texttt{01-getting-started.html} in the Files pane. This document is a
great way to record or share your work (you can also Knit to PDF or Word
documents).

\hypertarget{assigning-variables}{%
\subsection{Assigning variables}\label{assigning-variables}}

What's the difference between the code in this chunk:

\begin{Shaded}
\begin{Highlighting}[]
\FunctionTok{filter}\NormalTok{(mtcars, cyl }\SpecialCharTok{==} \DecValTok{4}\NormalTok{)}
\end{Highlighting}
\end{Shaded}

\begin{verbatim}
##                 mpg cyl  disp  hp drat    wt  qsec vs am gear carb
## Datsun 710     22.8   4 108.0  93 3.85 2.320 18.61  1  1    4    1
## Merc 240D      24.4   4 146.7  62 3.69 3.190 20.00  1  0    4    2
## Merc 230       22.8   4 140.8  95 3.92 3.150 22.90  1  0    4    2
## Fiat 128       32.4   4  78.7  66 4.08 2.200 19.47  1  1    4    1
## Honda Civic    30.4   4  75.7  52 4.93 1.615 18.52  1  1    4    2
## Toyota Corolla 33.9   4  71.1  65 4.22 1.835 19.90  1  1    4    1
## Toyota Corona  21.5   4 120.1  97 3.70 2.465 20.01  1  0    3    1
## Fiat X1-9      27.3   4  79.0  66 4.08 1.935 18.90  1  1    4    1
## Porsche 914-2  26.0   4 120.3  91 4.43 2.140 16.70  0  1    5    2
## Lotus Europa   30.4   4  95.1 113 3.77 1.513 16.90  1  1    5    2
## Volvo 142E     21.4   4 121.0 109 4.11 2.780 18.60  1  1    4    2
\end{verbatim}

And the code in this chunk?

\begin{Shaded}
\begin{Highlighting}[]
\NormalTok{four\_cyls }\OtherTok{\textless{}{-}} \FunctionTok{filter}\NormalTok{(mtcars, cyl }\SpecialCharTok{==} \DecValTok{4}\NormalTok{)}
\end{Highlighting}
\end{Shaded}

\hypertarget{functions}{%
\subsection{Functions}\label{functions}}

Look at the help page for \texttt{seq}

Add a chunk here that uses \texttt{seq()} to create a list of numbers
from 5 to 100, spaced by 5 (so 5, 10, 15, 20, etc.)

CHUNK HERE

\hypertarget{syntax-gone-wrong}{%
\subsection{Syntax gone wrong}\label{syntax-gone-wrong}}

\begin{Shaded}
\begin{Highlighting}[]
\FunctionTok{sd}\NormalTok{(}\FunctionTok{pull}\NormalTok{(}\AttributeTok{.data =}\NormalTok{ starwars, }\AttributeTok{var =}\NormalTok{ mass)}
\end{Highlighting}
\end{Shaded}

\begin{verbatim}
## Error: <text>:2:0: unexpected end of input
## 1: sd(pull(.data = starwars, var = mass)
##    ^
\end{verbatim}

\begin{Shaded}
\begin{Highlighting}[]
\NormalTok{my\_name }\OtherTok{\textless{}{-}} \StringTok{"Andrew\textquotesingle{}}
\end{Highlighting}
\end{Shaded}

\begin{verbatim}
## Error: <text>:1:12: unexpected INCOMPLETE_STRING
## 1: my_name <- "Andrew'
##                ^
\end{verbatim}

\begin{Shaded}
\begin{Highlighting}[]
\FunctionTok{pull}\NormalTok{(}\AttributeTok{.data =} \StringTok{"starwars"}\NormalTok{, }\AttributeTok{var =}\NormalTok{ height)}
\end{Highlighting}
\end{Shaded}

\begin{verbatim}
## Error in UseMethod("pull"): no applicable method for 'pull' applied to an object of class "character"
\end{verbatim}

\end{document}
